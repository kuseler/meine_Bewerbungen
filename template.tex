% FortySecondsCV LaTeX template
% Copyright © 2019-2022 René Wirnata <rene.wirnata@pandascience.net>
% Licensed under the 3-Clause BSD License. See LICENSE file for details.
%
% Please visit https://github.com/PandaScience/FortySecondsCV for the most
% recent version! For bugs or feature requests, please open a new issue on
% github.
%
% Contributors:
% https://github.com/PandaScience/FortySecondsCV/graphs/contributors
%
% Attributions
% ------------
% * fortysecondscv is based on the twentysecondcv class by Carmine Spagnuolo
%   (cspagnuolo@unisa.it), released under the MIT license and available under
%   https://github.com/spagnuolocarmine/TwentySecondsCurriculumVitae-LaTex
% * further attributions are indicated immediately before corresponding code


%-------------------------------------------------------------------------------
%                             ADDITIONAL PACKAGES
%-------------------------------------------------------------------------------
\documentclass[
	a4paper,
	% 9pt,
	% sidesectionsize=Large,
	% showframes,
	% vline=2.2em,
	% maincolor=cvgreen,
	% sidecolor=gray!50,
	% sidetextcolor=green,
	% sectioncolor=red,
	% subsectioncolor=orange,
	% itemtextcolor=black!80,
	sidebarwidth=0.34\paperwidth,
	% topbottommargin=0.03\paperheight,
	% leftrightmargin=20pt,
	% profilepicsize=4.5cm,
	% profilepicborderwidth=3.5pt,
	% profilepicstyle=profilecircle,
	% profilepiczoom=1.0,
	% profilepicxshift=0mm,
	% profilepicyshift=0mm,
	% profilepicrounding=1.0cm,
	% logowidth=4.5cm,
	% logospace=5mm,
	% logoposition=before,
	% sidebarplacement=right,
	% datecolwidth=0.22\textwidth,
]{fortysecondscv}

% fine tune line spacing
% \usepackage{setspace}
% \setstretch{1.1}

% improve word spacing and hyphenation
\usepackage{microtype}
\usepackage{ragged2e}

% uncomment in case you don't want any hyphenation
% \usepackage[none]{hyphenat}

% take care of proper font encoding
\ifxetexorluatex
	\usepackage{fontspec}
	\defaultfontfeatures{Ligatures=TeX}
	% \newfontfamily\headingfont[Path=fonts/]{segoeuib.ttf} % use local font
\else
	\usepackage[utf8]{inputenc}
	\usepackage[T1]{fontenc}
\fi

% use a sans serif font as default
\usepackage[sfdefault]{ClearSans}
% \usepackage[sfdefault]{noto}

% multi-language CV XeLaTeX and polyglossia (should also work with LuaLaTeX)
% NOTE: breaks \pointskill, \membership and some spacings
% \ifxetexorluatex
% 	\usepackage{polyglossia}
% 	\newfontfamily\arabicfontsf[Script=Arabic,Scale=1.5]{Amiri}
% 	\newfontfamily\englishfontsf{Clear Sans}
% 	\setmainfont{Amiri}
% 	\setdefaultlanguage{arabic}
% 	\setotherlanguage{english}
% \fi

% enable mathematical syntax for some symbols like \varnothing
\usepackage{amssymb}

\usepackage[ngerman]{babel}

% bubble diagram configuration
\usepackage{smartdiagram}
\smartdiagramset{
	% default font size is \large, so adjust to harmonize with sidebar layout
	bubble center node font = \footnotesize,
	bubble node font = \footnotesize,
	% default: 4cm/2.5cm; make minimum diameter relative to sidebar size
	bubble center node size = 0.4\sidebartextwidth,
	bubble node size = 0.25\sidebartextwidth,
	distance center/other bubbles = 1.5em,
	% set center bubble color
	bubble center node color = maincolor!70,
	% define the list of colors usable in the diagram
	set color list = {maincolor!10, maincolor!40,
	maincolor!20, maincolor!60, maincolor!35},
	% sets the opacity at which the bubbles are shown
	bubble fill opacity = 0.8,
}

%-------------------------------------------------------------------------------
%                            PERSONAL INFORMATION
%-------------------------------------------------------------------------------
%% mandatory information
% your name
\cvname{Kimi Müller}
% job title/career
\cvjobtitle{Schüler}

%% optional information
% profile picture
\cvprofilepic{pics/profile.png}
% logo picture
\cvlogopic{pics/logo_txt.png}

% NOTE: ordering in sidebar will mimic the following order
% date of birth
\cvbirthday{28.09.2003}
% short address/location, use \newline if more than 1 line is required
\cvaddress{Neue Straße 3 \newline 66909 Wahnwegen}
% phone number
\cvphone{+49 176 57685909}
% personal website
\cvsite{http://kimimueller.de}
% email address
\cvmail{kimi-mueller@protonmail.com}
% pgp key
%\cvkey{4096R/FF00FF00}{0xAABBCCDDFF00FF00}
% any other custom entry
%\cvcustomdata{\faFlag}{Chinese}

%-------------------------------------------------------------------------------
%                              SIDEBAR 1st PAGE
%-------------------------------------------------------------------------------
% add more profile sections to sidebar on first page
\addtofrontsidebar{
	% include gosquare national flags from https://github.com/gosquared/flags;
	% naming according to ISO 3166-1 alpha-2 country codes
	\graphicspath{{pics/flags/shiny/}}

	% social network accounts incl. proper hyperlinks
	\sidesection{Soziale Netzwerke}
		\begin{icontable}{1.5em}{1em}
			\social{\faGithub}
				{https://github.com/kuseler}
				{Github: kuseler}
			\social{\faXing}
				{https://www.xing.com/profile/Kimi_Mueller6}
				{Xing: Kimi Müller}
		\end{icontable}

	\sidesection{Languages}
		\pointskill{\flag{DE.png}}{Deutsch}{5}
		\pointskill{\flag{GB.png}}{Englisch}{4}
		\pointskill{\flag{FR.png}}{Französisch}{1}

	\sidesection{Hard Skills}
		\pointskill{\faPython}{Python3}{3}[5]
			\skill[1.8em]{\faCode}{Wettbewerbe, Challenges}
		\pointskill{\faPrint}{\LaTeX}{2}[5]
			\skill[1.8em]{\faSuperscript}{Dokumentation, Bewerbungen}
		\pointskill{\faCode}{GoLang}{2}[5]
			\skill[1.8em]{\faServer}{Hetzner API-Operator}
		\pointskill{\faLinux}{Linux}{2}[5]
			\skill[1.8em]{\faServer}{Eigener VPS, Arch Linux}
		\pointskill{\faDesktop}{Bash}{1}[5]
			\skill[1.8em]{\faCompress}{Grundkenntnisse}
		\pointskill{\faTable}{SQL (MySQL)}{1}[5]
			\skill[1.8em]{\faCompress}{Grundkenntnisse}

		

	
}


%-------------------------------------------------------------------------------
%                              SIDEBAR 2nd PAGE
%-------------------------------------------------------------------------------
\definecolor{pastelgreen}{HTML}{D7ECD9}
\definecolor{pastelpurple}{HTML}{D5D6EA}
\definecolor{pastelorange}{HTML}{F5D5CB}
\definecolor{pastelyellow}{HTML}{F6F6EB}

\addtobacksidebar{
	\graphicspath{{pics/flags/shiny/}}
	\sidesection{Soft Skills}
		\skill{\faBalanceScale}{Problemlösung}
		\skill{\faBalanceScale}{Teamfähigkeit}
	\sidesection{Über mich}
	\aboutme{
		Ich bin ein ambitioniertes junges Talent mit Codingerfahrung durch Wettbewerbe, Challenges und ein Praktikum. In meiner Freizeit spiele ich Handball und Schach, in letzterem bin ich Jugendleiter.
		}

% 	\sidesection{Diagrams}
% 	\begin{sidebarminipage}
% 		\chartlabel[pastelgreen]{Bubble}
% 		\chartlabel[pastelgreen]{Diagrams}
% 		\chartlabel[pastelpurple]{with}
% 		\chartlabel[pastelpurple]{proper}
% 		\chartlabel[pastelorange]{overflow}
% 		\chartlabel[pastelorange]{protection}
% 		\chartlabel[pastelyellow]{for}
% 		\chartlabel[pastelyellow]{labels}
% 	\end{sidebarminipage}

% 	\begin{figure}\centering
% 		\smartdiagram[bubble diagram]{
% 			\textcolor{white}{\textbf{Being a}} \\
% 			\textcolor{white}{\textbf{Panda}}, % center bubble
% 			\textcolor{black!90}{Eating},
% 			\textcolor{black!90}{Sleeping},
% 			\textcolor{black!90}{Rolling},
% 			\textcolor{black!90}{Playing},
% 			\textcolor{black!90}{Chilling}
% 		}
% 	\end{figure}
% 
% 	\chartlabel{Wheel Chart}
% 
% 	\wheelchart{3.7em}{2em}{%
% 	20/3em/maincolor!50/Chill,
% 	15/3em/maincolor!15/Play,
% 	30/4em/maincolor!40/Sleep,
% 	20/3em/maincolor!20/Eat
% 	}
% 
% 	\sidesection{Barskills}
% 	\barskill[1ex]{\faSkyatlas}{Wearing asian rice hats}{60}
% 	\barskill[2ex]{\faImage}{Playing Chess}{30}
% 	\barskill[3ex]{\faMusic}{Playing the bamboo flute}{50}

	\sidesection{Memberships}
	\begin{memberships}
		\membership[4em]{pics/logo.png}{PandaScience.net}
		\membership[4em]{pics/logo.png}{Some longer text spanning over more than
			only one line}
	\end{memberships}
}



%-------------------------------------------------------------------------------
%                         TABLE ENTRIES RIGHT COLUMN
%-------------------------------------------------------------------------------
\begin{document}

% Anschreiben

\begin{sidebar}
\begin{sidebarminipage}

		% optionally insert logo picture before profile
		\plotlogobefore

		% optionally insert profile picture
		\plotprofilepicture

		% optionally insert logo picture after profile
		\plotlogoafter

		\vspace{1ex}

		% name and job
		\nameandjob

		% personal information
		\vspace*{0.5em}
		\begin{icontable}[1.6]{1.7em}{0.4em}
			\personaldata
		\end{icontable}

		% user definitions\end{sidebarminipage}
\end{sidebarminipage}
\end{sidebar}

%  \newgeometry{
%  	top=0.5in,
%  	bottom=\topbottommargin,
%  	right=\leftrightmargin,
%  	left=\leftrightmargin
%  }

\cvsection{Bewerbung auf ein duales Studium} % ÄNDER

{\raggedright \large Sehr geehrte Damen und Herren,\\Ihre Ausschreibung für ein Duales Studium zum Bachelor Angewandte Informatik stieß bei mir auf großes Interesse. Die Skalierung, die ein Geschäftsvolumen von 2.980 Milliarden Euro ermöglicht, ist eine Herausforderung, die mich bei Ihnen besonders reizt.\\Ich habe großen Spaß an der Arbeit im Team, kann aber auch selbstständig, organisiert und engagiert. Seit meinem 16. Lebensjahr bin ich begeistert von Informatik und seit einem Jahr habe ich Interesse an Wirtschaft.\\In meiner Freizeit spiele ich Handball und Schach, in meinem Schachverein veranstalte ich das Jugendtraining. Des weiteren löse ich Coding-Challenges und nehme am Bundeswettbewerb Informatik teil.}

\newpage


% Ende Anschreiben

\makefrontsidebar

\cvsection{Berufserfahrung}
\begin{cvtable}[3]
	\cvitem{08.08.2022 -- 19.08.2022}{Praktikum als Software Developer}{System Vertrieb Alexander}{
	Grundlagen der Softwareentwicklung mit GoLang und Programmierung eines HTTP-Proxys. Anschließende Programmierung eines Operators zur Kreation und Löschung von virtuellen Servern und SSH-Keys. Zusätzlich dazu Konzeptionierung von Benchmarks zu CloudInit-Konfigurationen. Ein Ausblick auf Containerorchestrierung mit Kubernetes fand auch statt.}
	
	\cvitem{05.07.21 -- 15.07.21}{Praktikum als Elektroniker}{Pfalz-Alarm}{Prüfung und Wartung von Sicherheitssystemen, eigenständige Programmierung und Installation einer Brandmeldeanlage im Übungsbereich der Firma}
\end{cvtable}


\cvsection{Bildung}
\cvsubsection{Schulbildung}
\begin{cvtable}[]
	\cvitem{17.08.2020 -- 25.03.2023}{Abitur}{Siebenpfeiffer-Gymnasium Kusel}{
	Leistungsfächer: Informatik, Physik, Englisch
	}
	\\
	\cvitem{08.09.2014 -- 05.07.2020}{Mittlere Reife}{Siebenpfeiffer-Gymnasium Kusel}
		{Persönlicher Fokus auf MINT-Fächer}
\end{cvtable}


% \cvsubsection{Study}
% \begin{cvtable}[1.5]
% 	\cvitem{2006 -- 2008}{Master Studies Panda Science}{Panda Academy}
% 		{Focus: Advanced rice hat studies and nouveau rain-reflecting cover
% 		materials.}
% 	\cvitem{}{Master Thesis ($\varnothing\, 1,0$)}{Asian Rice Hat Institute}
% 		{Impact of solar radiation onto rice hat cover materials with special
% 		attention to water resistance.}
% 	\cvitem{2003 -- 2006}{Bachelor Studies PandaScience}{Panda Academy}
% 		{Focus: Bamboo morphology and its usage in different craftmanships.}
% 	\cvitem{}{Bachelor Thesis ($\varnothing\, 1,0$)}{Bamboo Institute}
% 		{The bambo flute: An underestimated instrument in orchestras?}
% \end{cvtable}
% 
% \cvsection{Publications}
% \begin{cvtable}
% 	\cvpubitem{Cooking: 100 recipes for lazy Pandas}{Me and My Panda Friends}
% 		{Panda's Culinary World}{2010}
% 	\cvpubitem{Pandastasia}{Still Me}{Bamboo Books Assoc.}{2005}
% \end{cvtable}
% 
% \cvsection{Awards}
% \begin{cvtable}
% 	\cvitem{2010 -- now}{Panda of the Year}{Panda World Forum}{}
% 	\cvitem{2005 -- now}{Face of World Wide Fund for Nature}{WWF}{}
% 	\cvitem{2000}{Winner of Bamboo Sprouts Eating Contest}{Bamboo Society}{}
% \end{cvtable}


\cvsection{Extra-Curricular Activities}
\begin{cvtable}
	\cvitemshort{Relaxing}{Master the fine art of relaxing everywhere}
	\cvitemshort{Music}{Playing the bamboo flute in the 1st Panda Orchestra}
	\cvitemshort{Education}{Teaching young pandas to be more panda-like}
\end{cvtable}


\newpage
\makebacksidebar
% \newgeometry{
% 	top=\topbottommargin,
% 	bottom=\topbottommargin,
% 	right=\leftrightmargin,
% 	left=\leftrightmargin
% }

\cvsection{section}
\cvsubsection{Subsection}
\begin{cvtable}
	\cvitem{<dates>}{<cv-item title>}{<location>}{<optional: description>}
\end{cvtable}

\cvsection{cvitem}
\cvsubsection{Multi-line with longer description}
\begin{cvtable}
	\cvitem{date}{Description}{location}{Some longer and more detailed
		description, that takes two lines of space instead of only one.}
	\cvitem{date}{Description}{location}{Some longer and more detailed
		description, that takes two lines of space instead of only one.}
	\cvitem{date}{Description}{location}{Some longer and more detailed
		description, that takes two lines of space instead of only one.}
\end{cvtable}

\cvsubsection{One-line without description}
\begin{cvtable}
	\cvitem{Award}{One-line description}{Sponsor}{}
	\cvitem{Award}{One-line description}{Sponsor}{}
	\cvitem{Award}{One-line description}{Sponsor}{}
\end{cvtable}

\cvsection{cvitemshort}
\cvsubsection{One-line}
\begin{cvtable}
	\cvitemshort{Key}{Some further description}
	\cvitemshort{Key}{Some further description}
	\cvitemshort{Key}{Some further description}
\end{cvtable}

\cvsubsection{Multi-line with longer description}
\begin{cvtable}
	\cvitemshort{Key}{Some further description. Can fill even more than
		only one single line while still keeping the correct indendation level.}
	\cvitemshort{Key}{Some further description. Can fill even more than
		only one single line while still keeping the correct indendation level.}
	\cvitemshort{Key}{Some further description. Can fill even more than
		only one single line while still keeping the correct indendation level.}
\end{cvtable}

\cvsection{cvpubitem}
\begin{cvtable}
	\cvpubitem{Publication title}{Authors}{Journal}{Year}
	\cvpubitem{Publication title}{Authors}{Journal}{Year}
	\cvpubitem{Publication title that is spanning over multiple lines and still
		does not look too bad}{Authors}{Journal}{Year}
\end{cvtable}


\cvsignature

\end{document}
