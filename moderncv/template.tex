%% start of file `template.tex'.
%% Copyright 2006-2015 Xavier Danaux (xdanaux@gmail.com), 2020-2022 moderncv maintainers (github.com/moderncv).
%
% This work may be distributed and/or modified under the
% conditions of the LaTeX Project Public License version 1.3c,
% available at http://www.latex-project.org/lppl/.



\documentclass[11pt,a4paper,sans]{moderncv}        % possible options include font size ('10pt', '11pt' and '12pt'), paper size ('a4paper', 'letterpaper', 'a5paper', 'legalpaper', 'executivepaper' and 'landscape') and font family ('sans' and 'roman')
\newcommand\blfootnote[1]{%
  \begingroup
  \renewcommand\thefootnote{}\footnote{#1}%
  \addtocounter{footnote}{-1}%
  \endgroup
}
\usepackage{changepage}% http://ctan.org/pkg/changepage
% moderncv themes
\moderncvstyle{classic}                             % style options are 'casual' (default), 'classic', 'banking', 'oldstyle' and 'fancy'
\moderncvcolor{blue}                               % color options 'black', 'blue' (default), 'burgundy', 'green', 'grey', 'orange', 'purple' and 'red'
%\renewcommand{\familydefault}{\sfdefault}         % to set the default font; use '\sfdefault' for the default sans serif font, '\rmdefault' for the default roman one, or any tex font name
%\nopagenumbers{}                                  % uncomment to suppress automatic page numbering for CVs longer than one page

\newcommand{\MYhref}[3][lightblue]{\href{#2}{\color{#1}{#3}}}%

% adjust the page margins
\usepackage[scale=0.8, top=2cm, bottom=1cm]{geometry}
%\setlength{\footskip}{149.60005pt}                 % depending on the amount of information in the footer, you need to change this value. comment this line out and set it to the size given in the warning
%\setlength{\hintscolumnwidth}{3cm}                % if you want to change the width of the column with the dates
%\setlength{\makecvheadnamewidth}{10cm}            % for the 'classic' style, if you want to force the width allocated to your name and avoid line breaks. be careful though, the length is normally calculated to avoid any overlap with your personal info; use this at your own typographical risks...

% font loading
% for luatex and xetex, do not use inputenc and fontenc
% see https://tex.stackexchange.com/a/496643
\ifxetexorluatex
  \usepackage{fontspec}
  \usepackage{unicode-math}
  \defaultfontfeatures{Ligatures=TeX}
  \setmainfont{Latin Modern Roman}
  \setsansfont{Latin Modern Sans}
  \setmonofont{Latin Modern Mono}
  \setmathfont{Latin Modern Math}
\else
  \usepackage[utf8]{inputenc}
  \usepackage[T1]{fontenc}
  \usepackage{lmodern}
\fi


% document language
\usepackage[german]{babel}  % FIXME: using spanish breaks moderncv

% personal data
\name{Kimi}{Müller}
\title{Lebenslauf}                               % optional, remove / comment the line if not wanted
\born{28.09.2003}                                 % optional, remove / comment the line if not wanted
\homepage{kimimueller.de}                         % optional, remove / comment the line if not wanted
\address{Neue Straße 3}{66909 Wahnwegen}% optional, remove / comment the line if not wanted; the "postcode city" and "country" arguments can be omitted or provided empty
\phone[mobile]{+49~176~576~85909}                   % optional, remove / comment the line if not wanted; the optional "type" of the phone can be "mobile" (default), "fixed" or "fax"
\email{kimi-mueller@proton.me}                               % optional, remove / comment the line if not wanted

% Social icons
\social[github]{kuseler}                              % optional, remove / comment the line if not wanted
\social[linkedin]{kimi-müller-197977242}
\social[xing]{Kimi\_Mueller6}                           % optional, remove / comment the line if not wanted


%\extrainfo{additional information}                 % optional, remove / comment the line if not wanted
\photo[64pt][0.4pt]{images/picture}                       % optional, remove / comment the line if not wanted; '64pt' is the height the picture must be resized to, 0.4pt is the thickness of the frame around it (put it to 0pt for no frame) and 'picture' is the name of the picture file
%\quote{Some quote}                                 % optional, remove / comment the line if not wanted

% bibliography adjustments (only useful if you make citations in your resume, or print a list of publications using BibTeX)
%   to show numerical labels in the bibliography (default is to show no labels)
%\makeatletter\renewcommand*{\bibliographyitemlabel}{\@biblabel{\arabic{enumiv}}}\makeatother
\renewcommand*{\bibliographyitemlabel}{[\arabic{enumiv}]}
%   to redefine the bibliography heading string ("Publications")
%\renewcommand{\refname}{Articles}

% bibliography with mutiple entries
%\usepackage{multibib}
%\newcites{book,misc}{{Books},{Others}}
%----------------------------------------------------------------------------------
%            content
%----------------------------------------------------------------------------------
\begin{document}

%-----       letter       ---------------------------------------------------------
% recipient data
\recipient{Firma}{Ort}
\date{\today}
\subject{Bewerbung um ein }
\opening{Sehr geehrte }
\closing{Mit freundlichen Grüßen,\\ \includegraphics[scale=0.4]{images/Unterschrift2.png}}
%\enclosure[Angehängt]{Lebenslauf, Abiturzeugnis, Arbeitszeugnis}          % use an optional argument to use a string other than "Enclosure", or redefine \enclname
\makelettertitle
%Nicht nur im Handball verfolge ich Ziele mit einer lösungsorientierten hands-on-Mentalität.
%Ich verfüge bereits über Codingerfahrung und auch in der Serveradministration habe ich Erfahrungen gesammelt. Dadurch bin ich bereits vom ersten Tag an eine wertvolle Ergänzung Ihres Teams.\
Ich freue mich auf Ihr Feedback und eine Einladung zu einem persönlichen Gespräch.\newline

\makeletterclosing



%\begin{CJK*}{UTF8}{gbsn}                          % to typeset your resume in Chinese using CJK
%-----       resume       ---------------------------------------------------------
\clearpage
\makecvtitle
% \section{Über mich}
% Ich bin ständig daran interessiert, mich zu entwickeln. Besonders Spaß machen mir knifflige Probleme und Herausforderungen, in Dinge hereinzuschauen und zu verstehen, wie sie funktionieren. Monotonie kann ich jedoch nur schwer ertragen. Der Austausch mit gleichgesinnten ist besonders interessant für mich. Besonders stolz bin ich auf den Server, den ich aufgesetzt habe, und
\section{Bildung}
\cventry{08/20\,-\,03/23}{Abitur}{Siebenpfeiffer-Gymnasium Kusel}{Kusel}{$\diameter=2,9$}{Leistungsfächer: Informatik, Physik, Englisch}
\cventry{09/14\,-\,07/20}{Mittlere Reife}{Siebenpfeiffer-Gymnasium Kusel}{Kusel}{}{Entwicklung von Interesse an MINT-Fächern}

\section{Berufserfahrung}
\subsection{Praktika}
\cventry{08/22\,-\,08/22}{Software-Developer}{System Vertrieb Alexander}{Wiesbaden}{}{Entwicklung von Software mit Golang, zweiwöchiges Praktikum}
%\cventry{year--year}{Job title}{Employer}{City}{}{General description no longer than 1--2 lines.\newline{}}
\begin{adjustwidth}{6.5em}{1cm}
Tätigkeiten:
\begin{itemize}
\setlength\itemsep{0.05em}
\item Erlernen der Grundlagen der Softwareentwicklung mit Golang
  \begin{itemize}
  \item Programmierung eines HTTP-Proxys
  \end{itemize}
\item Programmierung eines Operators für die Hetzner-API
  \begin{itemize}
  \item automatisierte Erstellung und Löschung von SSH-Keys und virtuellen Servern
  \end{itemize}
\item CloudInit-Konfiguration
  \begin{itemize}
  \item Grundlagen und Konzeptionierung von Benchmarks
  \end{itemize}
  \item Ausblick auf Containerorchestrierung mit Kubernetes
\end{itemize}
\end{adjustwidth}
\vspace{0.5\baselineskip}
\cventry{07/21\,-\,07/21}{Elektroniker}{Pfalz-Alarm GmbH}{Kaiserslautern}{}{Prüfung und Wartung von Sicherheitssystemen, zweiwöchiges Praktikum}
\begin{adjustwidth}{6.5em}{1cm}
Tätigkeiten:
\begin{itemize}
  \setlength\itemsep{0.05em}
  \item elektronische Prüfung von Brandmeldeanlagen
  \item eigenständige Installation und Programmierung einer Brandmeldeanlage im Übungsbereich der Firma
\end{itemize}
\end{adjustwidth}
%
% \section{Computer skills}
% \cvdoubleitem{category 1}{XXX, YYY, ZZZ}{category 4}{XXX, YYY, ZZZ}
% \cvdoubleitem{category 2}{XXX, YYY, ZZZ}{category 5}{XXX, YYY, ZZZ}
% \cvdoubleitem{category 3}{XXX, YYY, ZZZ}{category 6}{XXX, YYY, ZZZ}
\section{Kompetenzen}
%\cvitem{Technische Fähigkeiten}{}
%% Skill matrix as an alternative to rate one's skills, computer or other.

%% Adjusts width of skill matrix columns.
%% Usage \setcvskillcolumns[<width>][<factor>][<exp_width>]
%% <width>, <exp_width> should be lengths smaller than \textwidth, <factor> needs to be between 0 and 1.
%% Examples:
% \setcvskillcolumns[5em][][]%    adjust first column. Same as \setcvskillcolumns[5em]
% \setcvskillcolumns[][0.45][]%   adjust third (skill) column. Same as \setcvskillcolumns[][0.45]
% \setcvskillcolumns[][][\widthof{``Year''}]%     adjust fourth (years) column.
% \setcvskillcolumns[][0.45][\widthof{``Year''}]%
% \setcvskillcolumns[\widthof{``Languag''}][0.48][]
% \setcvskillcolumns[\widthof{``Languag''}]%

%% Adjusts width of legend columns. Usage \setcvskilllegendcolumns[<width>][<factor>]
%% <factor> needs to be between 0 and 1. <width> should be a length smaller than \textwidth
%% Examples:
% \setcvskilllegendcolumns[][0.45]
% \setcvskilllegendcolumns[\widthof{``Legend''}][0.453

% \setcvskilllegendcolumns[0ex][0.46]% this is usefull for the banking style

%% Add a legend if you are using \cvskill{<1-5>} command or \cvskillentry
%% Usage \cvskilllegend[*][<post_padding>][<first_level>][<second_level>][<third_level>][<fourth_level>][<fifth_level>]{<name>}
% \cvskilllegend % insert default legend without lines
%\cvskilllegend*[1em]{}% adjust post spacing
% %\cvskilllegend*{Legend}%  Alternatively add a description string
%% adjust the legend entries for other languages, here German
\cvskillplainlegend[0.09em][Grundkenntnisse][mittlere Kenntnisse][fortgeschrittene Kenntnisse][Expertenkenntnisse][Experte\,/\,Spezialist]{Legende}
%% Alternative legend style with the first three skill levels in one column
%% Usage \cvskillplainlegend[*][<post_padding>][<first_level>][<second_level>][<third_level>][<fourth_level>][<fifth_level>]{<name>}
\setcvskilllegendcolumns[][0.6]%  works for classic, casual, banking
% \setcvskilllegendcolumns[][0.55]%  works better for oldstyle and fancy
% \cvskillplainlegend{}
% \cvskillplainlegend[0.2em][Grundkenntnisse][Grundkenntnisse und eigene Erfahrung in Projekten][Umfangreiche Erfahrung in Projekten][Vertiefte Expertenkenntnisse][Experte/Guru]{Legende}

%% Add a head of the skill matrix table with descriptions.
%% Usage \cvskillhead[<post_padding>][<Level>][<Skill>][<Years>][<Comment>]%
%\cvskillhead[-0.1em]%   this inserts the standard legend in english and adjust padding
%% Adjust head of the skill matrix for other languages
\cvskillhead[0.25em][Level][F\"ahigkeit][Jahre][Bemerkung/Link zu Beispiel]

%% \cvskillentry[*][<post_padding>]{<skill_cathegory>}{<0-5>}{<skill_name>}{<years_of_experience>}{<comment>}%
%% Example usages:
\cvskillentry*{Coding:}{2}{Python3}{4}{\MYhref{https://www.codewars.com/users/kuseler/completed_solutions}{Coding-Challenges}, \MYhref{https://github.com/kuseler/HOI4-all-dlc-script/tree/master}{DLC editor}}
\cvskillentry{}{1}{Golang}{1}{{\MYhref{https://github.com/kuseler/hcloud-operate}{Hetzner-Operator}}, \MYhref{https://github.com/kuseler/costperuse-API}{API-Server}}
\cvskillentry{}{1}{shell (bash)}{1}{Grundlegende Automatisierung}
\cvskillentry*{OS:}{3}{Linux}{4}{\MYhref{https://kimimueller.de}{Betrieb eines eigenen Servers}}% notice the use of the starred command and the optional
\cvskillentry{}{1}{Docker}{2}{Grundkenntnisse}
\cvskillentry*{Office:}{3}{MS Excel}{6}{}
\cvskillentry{}{2}{\LaTeX}{3}{\MYhref{https://github.com/kuseler/meine_Bewerbungen/blob/main/moderncv/template.tex}{Bewerbungen}, Hausaufgaben}
\cvskillentry{}{2}{MS Word}{6}{}
\cvskillentry{}{2}{MS Powerpoint}{6}{}
\newpage
% \section{Sonstige}
%
% \cvlistdoubleitem{Microsoft Office-Suite}{git}
% \cvlistdoubleitem{Kontaktfreudig}{Lösungsorientiert}

%\cvskillentry*[1em]{Methods}{4}{SCRUM}{8}{SCRUM master for 5 years}
%% \cvskill{<0-5>} command
% \cvitem{\textbackslash{cvskill}:}{Skills can be visually expressed by the \textbackslash{cvskill} command, e.g. \cvskill{2}}




\section{Sprachen}
\cvitemwithcomment{Deutsch}{Muttersprache}{}
\cvitemwithcomment{Englisch}{B2/C1}{verhandlungssicher}
\cvitemwithcomment{Französisch}{B1}{selbstständige Sprachverwendung}

% \section{Hobbies und Soziales Engagement}
% \cvitemwithcomment{Schach}{SC Thallichtenberg Jugendleiter des Schachclub Thallichtenberg}
% % \cvitemwithcomment{a}{Social Media Verwalter der Herrenhandballmannschaft des TV Kusel}
%

\section{Hobbies und Soziales Engagement}
\cvitem{Handball}{Beim TV Kusel spiele ich Handball in der Position des Torhüters. Abseits des Felds agiere ich als Social-Media-Verwalter der Herrenmannschaft.}
\cvitem{Schach}{Im Schachclub Thallichtenberg spiele ich in der Bezirksliga West. Außerdem unterstütze ich meinen Verein als Jugendleiter.}
\cvitem{Coding}{In meiner Freizeit programmiere ich eigene Software-Projekte und absolviere Coding-Challenges.}


\section{Führerscheine}
\cvlistdoubleitem{Klasse B}{Klasse A2}


% Publications from a BibTeX file without multibib
%  for numerical labels: \renewcommand{\bibliographyitemlabel}{\@biblabel{\arabic{enumiv}}}% CONSIDER MERGING WITH PREAMBLE PART
%  to redefine the heading string ("Publications"): \renewcommand{\refname}{Articles}
% Publications from a BibTeX file using the multibib package
%\section{Publications}
%\nocitebook{book1,book2}
%\bibliographystylebook{plain}
%\bibliographybook{publications}                   % 'publications' is the name of a BibTeX file
%\nocitemisc{misc1,misc2,misc3}
%\bibliographystylemisc{plain}
%\bibliographymisc{publications}                   % 'publications' is the name of a BibTeX file

\blfootnote{Wenn sie Verbesserungsvorschläge oder Anregungen für diesen Lebenslauf haben, richten Sie diese bitte via E-Mail an mich.}

\clearpage
%\clearpage\end{CJK*}                              % if you are typesetting your resume in Chinese using CJK; the \clearpage is required for fancyhdr to work correctly with CJK, though it kills the page numbering by making \lastpage undefined
\end{document}


%% end of file `template.tex'.

